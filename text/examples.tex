\section{Introduction}
\newtheorem{theorem}{Theorem}

\blindtext \cite{IEEEtran}
\blindtext

\subsection{Table in paper}
\blindtext


\begin{table}[!thb]
\caption{Amazing table}
\centering
\begin{tabular}{lrrr} 
\toprule
Example table\\  
\midrule 
Column 1 & Column 2 & Column 3 & Column 4 	\\ 
\midrule 
Row 1	&	2	&	12,75	&	42	\\
Row 2	&	6	&	8,20	&	43	\\
Row 3	&	14	&	10,00	&	41	\\	 
\bottomrule
\end{tabular}
\end{table}

\subsection{Additionally IEEE-template-examples}
\blindtext
\begin{theorem}[Lorenz-Langmann]
This theorem had a funny assumption, which is here not described in more detail.
In the following the proof is described with one examples for the multi-row IEEE Equationarray environment:
\begin{IEEEproof}[Proof]
The proof is trivial. And rather it is none.
\begin{IEEEeqnarray}{rCl}
a	& = 	& b + c				\\
	& = 	& d + e + f + g + h
	+ i + j + k \nonumber		\\
	&		& +\> l + m + n + o		\\
	& = 	& p + q + r + s
\end{IEEEeqnarray}
\end{IEEEproof}
\end{theorem}

%Example for figure with two columns

\begin{figure*}[!t]
\centering
\subfloat[Case I]{\includegraphics[width=2.5in]{IEEE_Logo}
\label{fig_first_case}}
\hfil
\subfloat[Case II]{\includegraphics[width=2.5in]{IEEE_Logo}
\label{fig_second_case}}
\caption{Figure with two graphics distributed over two columns}
\label{fig_sim}
\end{figure*}


\subsubsection{Some text}
\blindtext


%Here again we have a \begin{table*} environment distributed over two columns!

\begin{table*}[!bt]
\caption{Amazing table with two columns with automatically column width, but far to little content}
\begin{tabularx}{1.0\textwidth}{XXXXXXXXX} \addlinespace \addlinespace \toprule 
%addlinespace is responsible for regular Vbox
Example table\\ 
 \midrule 
Column 1 & Column 2 & Column 3 & Column 4 	&	A	&	B	&	C	&	D	&	E		\\ 
\midrule 
Row 1	&	2	&	12,75	&	42	&	2	&	12,75	&	445645452	&	T T T	&	12,75	\\
Row 2	&	6	&	8,20	&	43	&	2	&	12,75	&	4456452	&	I J K	&	12,75		\\
Row 3	&	14	&	10,00	&	41	&	2	&	12,75	&	23242	&	M L N	&	12,75		\\	 
\bottomrule 
\end{tabularx}
\end{table*}

\subsection{Additionally text with references}
\Blindtext \blindtext \blindtext
\cite{Guenes+:2008TR}
\Blindtext

\begin{figure}[!htpb]
\centering
\includegraphics[width=0.8\columnwidth]{IEEE_Logo}
\caption{A common figure.}
\label{fig_sim2}
\end{figure}

\Blindtext